\documentclass[a4paper, 11pt]{article}

\usepackage[danish]{babel}
\usepackage[utf8]{inputenc}
%\usepackage{tgtermes}
%\usepackage{fouriernc}
\usepackage[T1]{fontenc}
\usepackage[margin=3cm]{geometry}

\usepackage{graphicx}
\usepackage{amssymb}
\usepackage{amsmath}
\usepackage{amsthm}
\usepackage{multicol}
\usepackage{xcolor}
\usepackage{wrapfig}
\usepackage{array} %skema

\usepackage{enumerate}
\usepackage[shortlabels]{enumitem}
\usepackage{verbatim}
\usepackage{hyperref}
\hypersetup{
    colorlinks=true,
    linkcolor=red,   
    urlcolor=red,
}
\newcommand{\N}{\mathbb{N}}
\newcommand{\Z}{\mathbb{Z}}
\newcommand{\Q}{\mathbb{Q}}
\newcommand{\R}{\mathbb{R}}

%Is defined to be equal to
\newcommand*{\defeq}{\mathrel{\vcenter{\baselineskip0.5ex \lineskiplimit0pt
                     \hbox{\scriptsize.}\hbox{\scriptsize.}}}%
                     =}
 
\title{{\large \textsc{Fysik eksamensprojekt}}}
\author{Cecilie Horshauge}
\date{\today}

\begin{document}
\maketitle
\section{Research}
\subsection{Fluid Dynamics}
Her er nogle eksempler på fluid dynamics i virkeligheden
\begin{itemize}
    \item Tornado
    \item Flyvemaskiner, holdt oppe af lufttryk på vingerne
    \item Væske i vores bræmser
    \item Vores krop
\end{itemize}
\subsection*{Densitet}
Densitet er masse per volumen og har SI enheden kg/\(\text{m}^3\).
\begin{equation}
    \label{densitet}
    \rho = \frac{m}{V}
\end{equation}
Vands densitet er som regel i nærheden af 1000 kg/\(\text{m}^3\). Hvor lufts densitet er cirka en faktor 1000 mindre.\\
Om væsker siger vi, at de er ikke-komprimerbare, deres densitet er forbliver altså næsten konstant. Gasser derimod kan komprimeres og kaldes derfor for komprimerbare.
\subsection*{Tryk}
I fluid mechanics defineres tryk som kraften en væske udfører per areal. SI enheden er \(\text{N}/\text{m}^2\), som kaldes pascal (Pa).
\begin{equation}
    \label{tryk}
    p=\frac{F}{A}
\end{equation}
Tryk er en skalar. I et givent punkt i en væske, er tryk fordelt ens i alle retninger. Det giver derfor ikke mening at tale om en retning. 
\end{document}