\documentclass[a4paper, 11pt]{article}

\usepackage[danish]{babel}
\usepackage[utf8]{inputenc}
\usepackage{macros}
%\usepackage{tgtermes}
%\usepackage{fouriernc}
\usepackage[T1]{fontenc}
\usepackage[margin=3cm]{geometry}

\usepackage{graphicx}
\usepackage{amssymb}
\usepackage{amsmath}
\usepackage{amsthm}
\usepackage{multicol}
\usepackage{xcolor}
\usepackage{wrapfig}
\usepackage{array} %skema

\usepackage{enumerate}
\usepackage[shortlabels]{enumitem}
\usepackage{verbatim}
\usepackage{hyperref}
\hypersetup{
    colorlinks=true,
    linkcolor=red,   
    urlcolor=red,
}
\newcommand{\N}{\mathbb{N}}
\newcommand{\Z}{\mathbb{Z}}
\newcommand{\Q}{\mathbb{Q}}
\newcommand{\R}{\mathbb{R}}

%Is defined to be equal to
\newcommand*{\defeq}{\mathrel{\vcenter{\baselineskip0.5ex \lineskiplimit0pt
                     \hbox{\scriptsize.}\hbox{\scriptsize.}}}%
                     =}
 
\title{{\large \textsc{HYDRODYNAMIK\\fysik eksamensprojekt}}}
\author{Cecilie Horshauge}
\date{\today}

\begin{document}
\maketitle
\section{Hypotese}
\section{Teori}
\subsection{Strømning i væsker}
Væskestrømninger opdeles overordnet set i to grupper: Turbulent og laminar strømning.\footnote{Hansen, O. W. 2008}
En turbulent strømning betyder at strømningen indeholder strømhvirvler, en strømning der ikke har det kaldes laminar.
Derudover siger vi, at en strømning er stationær, hvis hastigheden af en vilkårlig væskepartikel på et givent sted altid har samme retning og størrelse.
\myfigure{0.4}{Images/stromning.png}{Strømningslinjer}{Strømning}\\
På figur \ref{fig:Strømning} ses et rør med indtegnede strømningslinjer i. Der antages om væsken i røret, at den er inkompressibel, 
dvs. at densiteten \(\rho\) er ens overalt, hvilket gælder for væsker, men ikke for gasser.
Der er vist to tværsnit, \(A_1\) og \(A_2\), på figuren. På grund af rørets tykkelse vil væsken bevæge sig hurtigere ved \(A_2\) end ved \(A_1\).\\\\
Der ses på situationen hvor i tidsrummet \(dt\) strømmer væsken afstanden \(ds_1\) ved \(A_1\) gennem røret og ligeledes stykket \(ds_2\) ved \(A_2\).\\
Betragt nu disse udtryk for væskerumfangene.
\begin{center}
    \begin{tabular}{ l c l }
     \(dV_1=A_1ds_1\) & & \(dV_2=A_2ds_2\)
    \end{tabular}
\end{center}
Derudover er de to rumfang med sikkerhed ens, eftersom væsken er inkompressibel. Vi får da: \(V_1 = V_2\). Strækningen \(ds_1\) og \(ds_2\) kan også defineres ud fra hastigheden og tiden ved:
\begin{center}
    \begin{tabular}{ l c l }
     \(ds_1=v_1dt\) & & \(ds_2=v_2dt\)
    \end{tabular}
\end{center}
Derfor kan vi skrive
\[A_1v_1dt = A_2v_2dt\]
\begin{equation}
    A_1v_1=A_2v_2
\end{equation}
\subsection{Bernoullis lov}
Bernoullis lov beskriver sammenhængen mellem trykket \(p\) i et punkt af væsken med strømningshastigheden \(v\) i punktet. \footnote{Young. H. D. side 406-407}
\myfigure{0.4}{Images/bernoulli.png}{Visualisering af Bernoullis lov}{Bernoulli}\\
Betragt figur \ref{fig:Bernoulli} 
\subsection{Kontinuitetsligningen}
Massen af en fluid med en konstant strømning ændrer sig ikke. \footnote{Young. H. D. side 404-405}
Dette leder til et resultat kendt som Kontinuitetsligningen. Betragt to dele af en gennemstrømningsrør med strømning mellem to tværsnit, \(A_1\) og \(A_2\). Kald hastighederne af fluiderne ved tværsnitene for henholdsvis \(v_1\) og \(v_2\).
Der flyder ikke fluid ud af siderne af strømningsrøret og heller ikke ind af det. 
\subsection{Forventninger for forsøget}
\section{Fremgangsmåde}

\section{Resultater}
\section{Databehandling}
\subsection{Fluid Dynamics}
Her er nogle eksempler på fluid dynamics i virkeligheden
\begin{itemize}
    \item Tornado
    \item Flyvemaskiner, holdt oppe af lufttryk på vingerne
    \item Væske i vores bremser
    \item Vores krop
\end{itemize}
\subsection*{Densitet}
Densitet er masse per volumen og har SI enheden kg/\(\text{m}^3\).
\begin{equation}
    \label{densitet}
    \rho = \frac{m}{V}
\end{equation}
Vands densitet er som regel i nærheden af 1000 kg/\(\text{m}^3\). Hvor lufts densitet er cirka en faktor 1000 mindre.\\
Om væsker siger vi, at de er ikke-komprimerbare, deres densitet er forbliver altså næsten konstant. Gasser derimod kan komprimeres og kaldes derfor for komprimerbare.
\subsection*{Tryk}
I fluid mechanics defineres tryk som kraften en væske udfører per areal. SI enheden er \(\text{N}/\text{m}^2\), som kaldes pascal (Pa).
\begin{equation}
    \label{tryk}
    p=\frac{F}{A}
\end{equation}
Tryk er en skalar. I et givent punkt i en væske, er tryk fordelt ens i alle retninger. Det giver derfor ikke mening at tale om en retning. 
\subsection*{Hydrostatisk ligevægt}
Hydrostatisk ligevægt eller hydrostatisk balance (eng: Hydrostatic equilibrium) er den tilstand hvor summen af alle kræfter der virker på fluiden er lig 0.\\\\
Når der er tyngdekraft tilstede [...]
\subsection*{Pascals lov}
\end{document}